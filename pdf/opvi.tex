\documentclass[12pt,twoside]{article}
%\date{}   %uncommenting this erases the date
\usepackage{graphicx}
\usepackage{amsmath}
\usepackage{amssymb}
\usepackage{natbib}
\usepackage{verbatim}
\usepackage{floatpag}
\usepackage{subeqnarray}
\usepackage{mathrsfs}    %for special characters
\usepackage{cancel}  % to set terms in an equation to zero

\usepackage{hyperref}

\usepackage{amsthm}

\newtheorem{theorem}{Theorem}
\newtheorem{lemma}{Lemma}
\newtheorem{prop}{Proposition}


\setlength{\textheight}     {9.0in}
\setlength{\textwidth}      {6.5in}
\setlength{\oddsidemargin}  {0.0in}
\setlength{\evensidemargin} {0.0in}
\setlength{\topmargin}      {0.0in}
\setlength{\headheight}     {0.0in}
\setlength{\headsep}        {0.0in}
\setlength{\hoffset}        {0.0in}
\setlength{\voffset}        {0.0in}
\setlength{\parindent}      {0.0in}      %starting new line at extreme left

\graphicspath{{Figures/}}


\usepackage{bbm}
\begin{document}

\title{Operator Variational Inference}

\author{Raghav Singhal \& Saad Lahlou}

\maketitle

\section{Define Objective}

Note, for $\phi(z) = [ \phi_{1}(z_{1}) \dots \phi_{n}(z_n) ]$, we define the stein operator, $\mathcal{A}_{p(z|x)}$, as follows
\begin{align}
\mathbb{E}_{q_{\lambda}(z)}[ \mathcal{A}_{p(z | x)} \phi(z) ] &= \sum_{i=1}^{n}\mathbb{E}_{q_{\lambda}(z)}[ \mathcal{A}^{i}_{p(z_i | z_{-i}, x)} \phi_i(z_i) ]  \\
&= \sum_{i=1}^{n}\mathbb{E}_{q_{\lambda}(z_{-i})} \mathbb{E}_{q_{\lambda}(z_{i}| z_{-i})} [ \mathcal{A}^{i}_{p(z_i | z_{-i}, x)} \phi_i(z_i) ]
\end{align}
Now, for a fixed conditional, $z_i$, we ask the following question,

\begin{prop}\label{simplify}
Does $\mathbb{E}_{q_{\lambda}(z_i)}[ f(z_i) ] = \mathbb{E}_{q_{\lambda}(z_{-i})} \mathbb{E}_{q_{\lambda}(z_{i}| z_{-i})}[ f(z_i) ] $ for all $f$ continuos and bounded, here $f$ is a function of only $z_i$.
\end{prop}



\begin{proof}[Proof of Proposition 1]
Now, note that for any density $p(x,y) = p(x|y)p(y)$ and set $A_x \in \mathcal{B}(X) $, by Fubini's Theorem,
\begin{align*}
\int_{Y} \int_{X} \mathbb{I}_{A_x}(x) p(x|y)p(y) dx dy &= \int_{Y} \int_{X} \mathbb{I}_{A_x}(x) p(x,y) dy dx \\
&= \int_{A_{x}} \int_{Y} p(x,y) dy dx \text{  (check this step)}\\
&= \int_{A_{x}} (\int_{Y} p(x,y) dy ) dx  \\
& = \int_{X} \mathbb{I}_{A_x}(x) p(x) dx
\end{align*}
Now, for all $f(x) = \mathbb{I}_{A_x}(x)$ where $A_{x} \in \mathcal{B}(X)$, we just showed that,
\begin{align}
\mathbb{E}_{p(y)} \mathbb{E}_{p(x|y)}[ f(X) ] = \mathbb{E}_{p(x)}[ f(X)]
\end{align}

As $\forall f: X \rightarrow \mathbb{R} \text{ (Borel measurable \& Integrable)}$, there exists an increasing sequence of simple functions, s.t $f_{n} \rightarrow f \text{ and} f_{n} \leq f $, Proposition \ref{simplify} holds for all $f: X \rightarrow \mathbb{R} \text{ Borel measurable}$ and Integrable by the dominated convergence theorem.

\end{proof}

If Proposition \ref{simplify} holds, then equation 2 can be simplified to
\begin{align}
\mathbb{E}_{q_{\lambda}(z)}[ \mathcal{A}_{p(z | x)} \phi(z) ] &= \sum_{i=1}^{n}\mathbb{E}_{q_{\lambda}(z)}[ \mathcal{A}^{i}_{p(z_i | z_{-i}, x)} \phi_i(z_i) ]  \\
&= \sum_{i=1}^{n}\mathbb{E}_{q_{\lambda}(z_{-i})} \mathbb{E}_{q_{\lambda}(z_{i}| z_{-i})} [ \mathcal{A}^{i}_{p(z_i | z_{-i}, x)} \phi_i(z_i) ] \\
&= \sum_{i=1}^{n}\mathbb{E}_{q_{\lambda}(z_{i})}[ \mathcal{A}^{i}_{p(z_{i} | z_{-i}, x)} \phi_{i}(z_{i}) ]
\end{align}

Now, suppose $\phi(z) = [ \phi_{1}(z_{1}) \dots \phi_{n}(z_n) ]$, where $\phi_{i}(z_{i}) \in \mathcal{H}$ and $\mathcal{H}$ is a scalar-valued Reproducing Kernel Hilbert Space (RKHS).
\\
\\
Questions to tackle now
\begin{enumerate}
\item Whick Kernel should we select ? So we can detect non-convergence.
\item Is taking the supremum over $ \phi(z) \in \prod_{i=1}^{n} \mathcal{H}$ different from taking supremum over individual $\mathcal{H}$, so does the following hold
\begin{align*}
\sup_{\phi \in \prod_{i=1}^{n} \mathcal{H}} \mathbb{E}[\mathcal{A}\phi(Z)] = \sum_{i=1}^{n} \sup_{\phi_{i} \in \mathcal{H}} \mathbb{E}[\mathcal{A}^{i}\phi_{i}(Z_{i})]
\end{align*}
\end{enumerate}

\subsection{Attempt for item 2}

Note for a univariate distribution, the optimal $\phi^{*}$ is given by $\phi^{*}(y) = \mathbb{E}_{x \sim q}[ \mathcal{A}_{p}k(x,y) ]$. Now, note that
\begin{align*}
\sup_{\phi \in \prod_{i=1}^{n} \mathcal{H}} \mathbb{E}_{q_{\lambda}(z)}[ \mathcal{A}_{p(z | x)} \phi(z) ] &= \sup_{\phi \in \prod_{i=1}^{n} \mathcal{H}}\sum_{i=1}^{n}\mathbb{E}_{q_{\lambda}(z_{i})}[ \mathcal{A}^{i}_{p(z_{i} | z_{-i}, x)} \phi_{i}(z_{i}) ] \\
& \leq \sum_{i=1}^{n} \sup_{\phi_{i} \in \mathcal{H}} \mathbb{E}_{q_{\lambda}(z_{i})}[ \mathcal{A}^{i}_{p(z_{i} | z_{-i}, x)} \phi_{i}(z_{i}) ]
\end{align*}

Now, all we have to show is that the reverse inequality holds true,
\begin{align*}
\sum_{i=1}^{n} \sup_{\phi_{i} \in \mathcal{H}} \mathbb{E}_{q_{\lambda}(z_{i})}[ \mathcal{A}^{i}_{p(z_{i} | z_{-i}, x)} \phi_{i}(z_{i}) ] \leq \sup_{\phi \in \prod_{i=1}^{n} \mathcal{H}}\sum_{i=1}^{n}\mathbb{E}_{q_{\lambda}(z_{i})}[ \mathcal{A}^{i}_{p(z_{i} | z_{-i}, x)} \phi_{i}(z_{i}) ]
\end{align*}

\textbf{This is probably True. Show Rigorously !}

\subsection{Final Objective}
Assuming the above holds, we take
\begin{align}
\sup_{\phi \in \prod_{i=1}^{n} \mathcal{H}} \mathbb{E}_{q_{\lambda}(z)}[ \mathcal{A}_{p(z | x)} \phi(z) ] &= \sup_{\phi \in \prod_{i=1}^{n} \mathcal{H}}\sum_{i=1}^{n}\mathbb{E}_{q_{\lambda}(z_{i})}[ \mathcal{A}^{i}_{p(z_{i} | z_{-i}, x)} \phi_{i}(z_{i}) ] \\
&= \sum_{i=1}^{n} \sup_{\phi_{i} \in \mathcal{H}} \mathbb{E}_{q_{\lambda}(z_{i})}[ \mathcal{A}^{i}_{p(z_{i} | z_{-i}, x)} \phi_{i}(z_{i}) ]
\end{align}

Now, note the optimal test function is $\frac{\phi_{i}^{*}()}{||\phi_{i}^{*}||_{\mathcal{H}}}$, where $\phi_{i}^{*}()$ is given by
\begin{align*}
\phi_{i}^{*}(y) &= \mathbb{E}_{z^{*}_{i} \sim q_{\lambda}(z^{*}_i)}[ \mathcal{A}^{i}_{p(z^{*}_{i}|z^{*}_{-i}, x)}k(z^{*}_{i}, y) ] \\
&= \mathbb{E}_{z^{*}_{i} \sim q_{\lambda}(z^{*}_i)} [ k(z^{*}_i, y) \nabla_{z^{*}_i} \log p(z^{*}_i|z^{*}_{i-1},x) + \nabla_{z^{*}_i} k(z^{*}_i, y)]
\end{align*}
which implies that
\begin{align*}
\sup_{\phi \in \prod_{i=1}^{n} \mathcal{H}} \mathbb{E}_{q_{\lambda}(z)}[ \mathcal{A}_{p(z | x)} \phi(z) ] &= \sum_{i=1}^{n} \sup_{\phi_{i} \in \mathcal{H}} \mathbb{E}_{q_{\lambda}(z_{i})}[ \mathcal{A}^{i}_{p(z_{i} | z_{-i}, x)} \phi_{i}(z_{i}) ] \\
&= \sum_{i=1}^{n} \mathbb{E}_{q_{\lambda}(z_{i})}[ \mathcal{A}^{i}_{p(z_{i} | z_{-i}, x)} \phi^{*}_{i}(z_{i}) ] \\
&= \sum_{i=1}^{n} \mathbb{E}_{q_{\lambda}(z_{i})}\big[ \mathcal{A}^{i}_{p(z_{i} | z_{-i}, x)} \mathbb{E}_{z^{*}_{i} \sim q_{\lambda}(z^{*}_i)}[ \mathcal{A}^{i}_{p(z^{*}_{i}|z^{*}_{-i}, x)}k(z^{*}_{i}, z_i) ] \big]
\end{align*}
Now, for one conditional the final objective is as follows:
\begin{align}
\mathbb{E}_{q_{\lambda}(z_{i})}[ \mathcal{A}^{i}_{p(z_{i} | z_{-i}, x)} \phi^{*}_{i}(z_{i}) ]  &= \mathbb{E}_{q_{\lambda}(z_{i})}[ \phi_{i}^{*}(z_{i}) \nabla_{z_{i}}\log p(z_{i}|z_{-i},x) + \nabla_{z_{i}} \phi_{i}^{*}(z_{i})]
\end{align}

\section{Take Unbiased Gradient}

Now, we write the gradient of the Langevin-Stein Operator with respect to the variational parameters,
\begin{align*}
\nabla_{\lambda} \mathbb{E}_{q_{\lambda}(z_i)}[\mathcal{A}_{p(z_{i}|z_{-i},x)} \phi_i^{*}(z_i)] &=  \nabla_{\lambda} \mathbb{E}_{q_{\lambda}(z_{i})}[ \phi_{i}^{*}(z_{i}) \nabla_{z_{i}}\log p(z_{i}|z_{-i},x) + \nabla_{z_{i}} \phi_{i}^{*}(z_{i})] \\
&= \nabla_{\lambda} \mathbb{E}_{q_{\lambda}(z_{i})}[ \phi_{i}^{*}(z_{i}) \nabla_{z_{i}}\log p(z_{i}|z_{-i},x)] + \nabla_{\lambda} \mathbb{E}_{q_{\lambda}(z_{i})}[\nabla_{z_{i}} \phi_{i}^{*}(z_{i})] \\
& := \mathcal{L}_{\lambda} + \mathcal{I}_{\lambda}
\end{align*}
Now, note that integrals of the form above can be evaluated as follows:
\begin{align*}
\mathcal{J}_{\lambda} &= \nabla_{\lambda} \int_{z_{i} \in \Omega_{i}} \int_{z^{*}_{i} \in \Omega_{i}} q_{\lambda}(z_i) q_{\lambda}(z^{*}_i) f(z_i, z^{*}_i) dz_{i} dz_{i}^{*} \\
&= \int_{z_{i} \in \Omega_{i}} \int_{z^{*}_{i} \in \Omega_{i}} \nabla_{\lambda} \big( q_{\lambda}(z_i) q_{\lambda}(z^{*}_i) \big) f(z_i, z^{*}_i) dz_{i} dz_{i}^{*} \\
&=  \int_{z_{i} \in \Omega_{i}} \int_{z^{*}_{i} \in \Omega_{i}} q_{\lambda}(z_i) \nabla_{\lambda} q_{\lambda}(z^{*}_{i}) f(z_i, z^{*}_i) dz_{i} dz_{i}^{*} + \int_{z_{i} \in \Omega_{i}} \int_{z^{*}_{i} \in \Omega_{i}} q_{\lambda}(z_{i}^{*}) \nabla_{\lambda} q_{\lambda}(z_{i}) f(z_i, z^{*}_i) dz_{i} dz_{i}^{*} \\
&= \int_{z_{i} \in \Omega_{i}} \int_{z^{*}_{i} \in \Omega_{i}} q_{\lambda}(z_i) q_{\lambda}(z^{*}_i) f(z_i, z^{*}_i) \nabla_{\lambda}( \log q_{\lambda}(z_i) + \log q_{\lambda}(z^{*}_{i})) dz_{i} dz_{i}^{*}
\end{align*}
which implies that
\begin{align}
\mathcal{I}_{\lambda} &= \nabla_{\lambda}\mathbb{E}_{q_{\lambda}(z_{i})} \mathbb{E}_{q_{\lambda}(z^{*}_{i})}[ \mathcal{A}^{i}_{p(z^{*}_{i}|z^{*}_{-i}, x)} \nabla_{z_{i}} k(z^{*}_i, z_i) ] \\
&= \mathbb{E}_{q_{\lambda}(z_{i})} \mathbb{E}_{q_{\lambda}(z^{*}_{i})} [  \nabla_{\lambda}(\log q_{\lambda}(z_i) + \log q_{\lambda}(z^{*}_{i}) ) \mathcal{A}^{i}_{p(z^{*}_{i}|z^{*}_{-i}, x)} \nabla_{z_{i}} k(z^{*}_i, z_i) ]
\end{align}
And we can similarly evaluate $\mathcal{L}_{\lambda}$.
\end{document}
